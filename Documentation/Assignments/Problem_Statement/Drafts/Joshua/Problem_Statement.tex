\documentclass[onecolumn, draftclsnofoot,10pt, compsoc]{IEEEtran}
\usepackage{graphicx}
\usepackage{url}
\usepackage{setspace}

\usepackage{geometry}
\geometry{textheight=9.5in, textwidth=7in}

% 1. Fill in these details
\def \CapstoneTeamName{30k CS Avionics}
\def \CapstoneTeamNumber{41}
\def \GroupMemberOne{Joshua Novak}
\def \GroupMemberTwo{Allison Sladek}
\def \GroupMemberThree{Levi Willmeth}
\def \CapstoneProjectName{30k Rocket Spaceport America}
\def \CapstoneSponsorCompany{Mechanical Engineering, Oregon State University}
\def \CapstoneSponsorPerson{Nancy Squires}

% 2. Uncomment the appropriate line below so that the document type works
\def \DocType{Problem Statement
				%Requirements Document
				%Technology Review
				%Design Document
				%Progress Report
				}

\newcommand{\NameSigPair}[1]{\par
\makebox[2.75in][r]{#1} \hfil 	\makebox[3.25in]{\makebox[2.25in]{\hrulefill} \hfill		\makebox[.75in]{\hrulefill}}
\par\vspace{-12pt} \textit{\tiny\noindent
\makebox[2.75in]{} \hfil		\makebox[3.25in]{\makebox[2.25in][r]{Signature} \hfill	\makebox[.75in][r]{Date}}}}
% 3. If the document is not to be signed, uncomment the RENEWcommand below
%\renewcommand{\NameSigPair}[1]{#1}

%%%%%%%%%%%%%%%%%%%%%%%%%%%%%%%%%%%%%%%
\begin{document}
\begin{titlepage}
    \pagenumbering{gobble}
    \begin{singlespace}
    	%\includegraphics[height=4cm]{coe_v_spot1}
        \hfill
        % 4. If you have a logo, use this includegraphics command to put it on the coversheet.
        %\includegraphics[height=4cm]{CompanyLogo}
        \par\vspace{.2in}
        \centering
        \scshape{
            \huge CS Capstone \DocType \par
            {\large\today}\par
            \vspace{.5in}
            \textbf{\Huge\CapstoneProjectName}\par
            \vfill
            {\large Prepared for}\par
            \Huge \CapstoneSponsorCompany\par
            \vspace{5pt}
            {\Large\NameSigPair{\CapstoneSponsorPerson}\par}
            {\large Prepared by }\par
            Group\CapstoneTeamNumber\par
            % 5. comment out the line below this one if you do not wish to name your team
            \CapstoneTeamName\par
            \vspace{5pt}
            {\Large
                \NameSigPair{\GroupMemberOne}\par
                \NameSigPair{\GroupMemberTwo}\par
                \NameSigPair{\GroupMemberThree}\par
            }
            \vspace{20pt}
        }
        \begin{abstract}
        % 6. Fill in your abstract
        	The document will explain the goal of our project.
					This includes and explanation of what the client has requested, our proposed solution, and a detailed description of what constitutes success on our part.
				\end{abstract}
    \end{singlespace}
\end{titlepage}
\newpage
\pagenumbering{arabic}
%\tableofcontents
% 7. uncomment this (if applicable). Consider adding a page break.
%\listoffigures
%\listoftables
\clearpage

% 8. now you write!
\section{Basic Problem}
Our client wants a rocket that will reach 30k feet, can be slowed to descend at that altitude, and carries a payload.
The payload needs to undergo 8-12 seconds of weightlessness during which a zero-g experiment will be performed inside of it.
Both the payload and the rocket must be recovered.
Data from the experiment must be translated into useful information.

\section{Our Role}
As the CS team on this project, our responsibility is first to ensure that live telemetric data from the payload and rocket is being received and translated into useful information for tracking and recovering them.
Secondly, we must make sure that the data from the zero-g experiment is compiled into a readable and usable format.
Finally, we are to assist the other teams on the project with whatever tasks require our expertise.
An example of this would be the avionics and recovery team, which will require our assistance with programming the altimeters.

\section{Solutions}
To deal with the first problem we will need to meet with the team that will handle sending telemetry data from the rocket and payload, which is made up of Electronic Engineers.
The EE team and our team will need to discuss and decide on what format we want the data to be sent in, how it will be sent, and how it will be received.
Based on past projects it seems likely the information will be sent over a HAM radio, in which case one of our team members will likely need to acquire a HAM radio license.
We will also need to translate this data into a readable format.
For the second problem we will need to work with the EE team so that we will know how they plan to store the results of the experiment.
From this we will be able to gather how to translate that data into readable information, likely in the format of graphs.
For the final problem, we will need to maintain close contact with all other teams to see where they will require programming experience to complete a task, such as with the altimeters.
To accomplish this we will make sure to attend weekly meetings where all teams discuss their plans.

\section{Metrics for Success}
There are several tasks that must be completed for our project to be considered a success.
Most importantly, we need to have a successful test launch of the rocket and a successful test of the payload.
Failure to translate live telemetric data into readable information for these tests would constitute a failure on our part.
We will also need to be able to translate data from the zero-g experiment into readable information.
\end{document}
