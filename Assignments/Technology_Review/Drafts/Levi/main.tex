\documentclass[onecolumn, draftclsnofoot, 10pt, compsoc]{IEEEtran}
\usepackage{graphicx}
\usepackage{url}
\usepackage{setspace}
\usepackage{geometry}

\geometry{textheight=9.5in, textwidth=7in}

% 1. Fill in these details
\def \CapstoneTeamName{			Team 41}
\def \CapstoneTeamNumber{		41}
\def \GroupName{				30k CS Avionics}
\def \GroupMemberOne{			Joshua Novak}
\def \GroupMemberTwo{			Allison Sladek}
\def \GroupMemberThree{			Levi Willmeth}
\def \CapstoneProjectName{		30K Rocket Spaceport America}
\def \CapstoneSponsorCompany{	Oregon State University}
\def \CapstoneSponsorPerson{	Dr. Nancy Squires}

% 2. Uncomment the appropriate line below so that the document type works
\def \DocType{		%Problem Statement
					%Requirements Document
					Technology Review
					%Design Document
					%Progress Report
				}
			
\newcommand{\NameSigPair}[1]{
	\par
	\makebox[2.75in][r]{#1} \hfill
	\makebox[3.25in]{\makebox[2.25in]{\hrulefill} \hfill \makebox[.75in]{\hrulefill}}
	\par\vspace{-12pt}
	\textit{
		\tiny\noindent \makebox[2.75in]{} \hfill
		\makebox[3.25in]{\makebox[2.25in][r]{Signature} \hfill \makebox[.75in][r]{Date}}
	}
}
% 3. If the document is not to be signed, uncomment the RENEWcommand below
%\renewcommand{\NameSigPair}[1]{#1}
% \renewcommand{\thesubsubsection}{\thesection.\alph{subsubsection}}

%%%%%%%%%%%%%%%%%%%%%%%%%%%%%%%%%%%%%%%
\begin{document}
\begin{titlepage}
    \pagenumbering{gobble}
    \begin{singlespace}
    	%\includegraphics[height=4cm]{coe_v_spot1}
        \hfill 
        % 4. If you have a logo, use this includegraphics command to put it on the coversheet.
        %\includegraphics[height=4cm]{CompanyLogo}   
        \par\vspace{.2in}
        \centering
        \scshape{
            \huge CS Capstone \DocType \par
            {\large\today}\par
            \vspace{.5in}
            \textbf{\Huge\CapstoneProjectName}\par
            \vfill
            {\large Prepared for}\par
%             \Huge \CapstoneSponsorCompany\par
            \vspace{5pt}
            {\Large\NameSigPair{\CapstoneSponsorPerson}\par}
            {\large Prepared by }\par
           	\GroupName\par
            % 5. comment out the line below this one if you do not wish to name your team
            \CapstoneTeamName\par
            \vspace{5pt}
            {\Large
%                 \NameSigPair{\GroupMemberOne}\par
%                 \NameSigPair{\GroupMemberTwo}\par
                \NameSigPair{\GroupMemberThree}\par
            }
            \vspace{20pt}
        }
    \end{singlespace}
    
    \begin{abstract}
    	% 6. Fill in your abstract
        This is a technology review for several telemetry transmitters, methods of decoding APRS packets, and database solutions to organize and store flight data.
	\end{abstract}
\end{titlepage}
\newpage

\pagenumbering{arabic}

\section*{Revision History}
\begin{center}
	\begin{tabular*}{1\linewidth}{@{\extracolsep{\fill}}|c|c|c|c|}
        \hline
	    Name & Date & Reason For Changes & Version\\
        \hline
        Levi Willmeth&11/3/17&Initial document draft&0.1\\
        \hline
        & & &\\
        \hline
    \end{tabular*}
\end{center}

\tableofcontents
% 7. uncomment this (if applicable). Consider adding a page break.
%\listoffigures
%\listoftables
\section{Project Overview}
\subsection{Introduction}
Our project is to design, build, and test software that will fly on board the Oregon State University's entry to the Spaceport America Cup's 30k Challenge.  The Spaceport America Cup is an international engineering competition to design, build, and fly a student-made rocket to 30,000 feet.  The competition is scored on several criteria including software components like flight avionics, recording and displaying telemetry, and using a scientific research payload.

\subsection{My personal role in the team}
Our team has many members, including at least 12 mechanical engineers, 3 electrical engineers, and 3 computer science students, as well as several faculty and industry mentors.  My role in the team is to work with my computer science subteam to design, build, and test the ground station software, as well as the flight software that will control avionics for the rocket, and a scientific payload that will be ejected from the rocket at apogee.

Together, our sub team of computer scientists will work together to divide the larger project into 4 groups of software: rocket avionics, payload avionics, parsing data, and displaying data.  Our goal is to work together so that all of us contribute to each piece, even though one of us may be called on to take the lead on different pieces at different times.  Personally, my primary interests lie in the rocket and payload avionics components, but I also have experience working with parsing and graphing data.

\section{Transmitting telemetry data}

\subsection{Overview of telemetry}
In this context, telemetry is data collected from the rocket during flight.  This data will be used to calculate the rocket's maximum altitude, as well as locate the rocket after the flight.  If something goes wrong during the flight and the rocket is not recovered, telemetry can be used to determine what went wrong so that future flights may avoid the same problem.

For this project, we will use a commercial, off the shelf telemetry module to transmit data from the rocket to the ground.  There are several commercial products that offer this capability.  This section will outline 3 options and explain which device our team will use.

\subsection{Bigredbee BeeLine GPS}
The BeeLine GPS unit is a self contained GPS and telemetry transmitter that operates on the 70cm radio band.  It is 1.25" x 3", weighs about 2 ounces, and commonly uses a 35cm long antenna.  The BeeLine GPS can also record up to 2 hours of GPS data internally.  The BeeLine GPS can log and transmit the following telemetry fields at 1Hz intervals at 100mW of power:

\begin{itemize}
	\item Latitude
    \item Longitude
    \item Altitude
    \item Second/minute/hour/day/month/year timestamp. (probably)
    \item Number of GPS satellites visible (measure of location confidence) (probably)
\end{itemize}

The BeeLine GPS transmits an audio signal that needs to be decoded on the ground, using a terminal node controller (TNC).  There are several devices that can accomplish this, including using the sound card on a computer.  The price varies by vendor, but most hardware TNC's cost over \$80.  Software TNC's are limited to one signal per sound card, which makes exact pricing difficult to determine.  Using a dedicated raspberry pi zero with USB sound card, the cost to process APRS packets with a software TNC should be around \$20 per signal.

It is unknown how many signals can be processed by a single computer, but it seems reasonable to assume that processing additional signals will eventually begin to reduce the rate of successful packet captures.  In other words, processing each signal with a dedicated computer may add redundancy and performance.

\subsection{Altus Metrum TeleMega}
The TeleMega is a self contained GPS and telemetry transmitter on the 70cm radio band, that can also function as a flight computer.  It measures 1.25" x 3.25" and often uses a 35cm long antenna. The data is sent as an audio packet that is decoded by the receiver.  The TeleMega can store several hours of data on board, and transmits different types of telemetry at custom intervals, using 40mW of power.

The GPS location data is transmitted at 1Hz intervals and include:
\begin{itemize}
	\item Latitude
    \item Longitude
    \item Altitude
    \item Ground speed (cm/s)
    \item Ascent rate (cm/s)
    \item Bearing (direction of travel)
    \item Number of GPS satellites visible (measure of location confidence)
    \item Second/minute/hour/day/month/year timestamp.
\end{itemize}

Additionally, the TeleMega also sends a sensor status packet at 10Hz intervals, which include:
\begin{itemize}
	\item Euler $\beta$, which is the angle in degrees from vertical.
    \item 100g accelerometer on Z axis
    \item 3 axis, 16g accelerometer
    \item 3 axis, 2000 deg/sec gyroscope
    \item 3 axis magnetometer
    \item Pressure (Pa * 10)
    \item Temperature (C)
\end{itemize}

Additionally, the TeleMega also sends a flight status packet at 5Hz intervals, which include:
\begin{itemize}
	\item Device ID
    \item Flight number
    \item Configuration version number
    \item Apogee deploy delay (s)
    \item Main parachute deploy (m)
    \item Radio operator identifier
    \item Flight state
    \item Battery voltage
    \item Pyro battery voltage
    \item Calculated height (based on pressure recorded before launch)
\end{itemize}

The TeleMega data packets are formatted in 32 bit packets to reduce errors, and are designed to be received and decoded by a TeleDongle ground station.  The TeleDongle is an additional cost but also provides a convenient serial output over USB.  This is important later, because it means a single computer can simultaneously receive multiple signals from different transmitters.

The TeleMega costs around \$280 and offers a 10\% student discount.  The TeleDongle is another ~\$100 per signal.

\end{document}