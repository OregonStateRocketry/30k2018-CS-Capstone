\documentclass[onecolumn, draftclsnofoot, 10pt, compsoc]{IEEEtran}
\usepackage{graphicx}
\usepackage{url}
\usepackage{setspace}
\usepackage{geometry}

\geometry{textheight=9.5in, textwidth=7in}

% 1. Fill in these details
\def \CapstoneTeamName{			Team Name TBD}
\def \CapstoneTeamNumber{		41}
\def \GroupName{				30k CS Avionics}
\def \GroupMemberOne{			Joshua Novak}
\def \GroupMemberTwo{			Allison Sladek}
\def \GroupMemberThree{			Levi Willmeth}
\def \CapstoneProjectName{		30K Rocket Spaceport America}
\def \CapstoneSponsorCompany{	Oregon State University}
\def \CapstoneSponsorPerson{	Dr. Nancy Squires}

% 2. Uncomment the appropriate line below so that the document type works
\def \DocType{		%Problem Statement
					Requirements Document
					%Technology Review
					%Design Document
					%Progress Report
				}
			
\newcommand{\NameSigPair}[1]{
	\par
	\makebox[2.75in][r]{#1} \hfill
	\makebox[3.25in]{\makebox[2.25in]{\hrulefill} \hfill \makebox[.75in]{\hrulefill}}
	\par\vspace{-12pt}
	\textit{
		\tiny\noindent \makebox[2.75in]{} \hfill
		\makebox[3.25in]{\makebox[2.25in][r]{Signature} \hfill \makebox[.75in][r]{Date}}
	}
}
% 3. If the document is not to be signed, uncomment the RENEWcommand below
%\renewcommand{\NameSigPair}[1]{#1}
% \renewcommand{\thesubsubsection}{\thesection.\alph{subsubsection}}

%%%%%%%%%%%%%%%%%%%%%%%%%%%%%%%%%%%%%%%
\begin{document}
\begin{titlepage}
    \pagenumbering{gobble}
    \begin{singlespace}
    	%\includegraphics[height=4cm]{coe_v_spot1}
        \hfill 
        % 4. If you have a logo, use this includegraphics command to put it on the coversheet.
        %\includegraphics[height=4cm]{CompanyLogo}   
        \par\vspace{.2in}
        \centering
        \scshape{
            \huge CS Capstone \DocType \par
            {\large\today}\par
            \vspace{.5in}
            \textbf{\Huge\CapstoneProjectName}\par
            \vfill
            {\large Prepared for}\par
            %\Huge \CapstoneSponsorCompany\par
            \vspace{5pt}
            {\Large\NameSigPair{\CapstoneSponsorPerson}\par}
            {\large Prepared by }\par
           	\GroupName\par
            % 5. comment out the line below this one if you do not wish to name your team
            %\CapstoneTeamName\par
            \vspace{5pt}
            {\Large
                \NameSigPair{\GroupMemberOne}\par
                \NameSigPair{\GroupMemberTwo}\par
                \NameSigPair{\GroupMemberThree}\par
            }
            \vspace{20pt}
        }
    \end{singlespace}
    
    \begin{abstract}
    	% 6. Fill in your abstract
        Client requirements for the computer science portion of the Oregon State University's entry into the 30k Spaceport America Cup in 2018.  The software components outlined in this requirements document control the rocket avionics, record and display live telemetry from the rocket, and record and display results from the scientific payload.
	\end{abstract}
\end{titlepage}
\newpage

\pagenumbering{arabic}
\tableofcontents
% 7. uncomment this (if applicable). Consider adding a page break.
%\listoffigures
%\listoftables
\clearpage

\section*{Revision History}
\begin{center}
	\begin{tabular*}{1\linewidth}{@{\extracolsep{\fill}}|c|c|c|c|}
        \hline
	    Name & Date & Reason For Changes & Version\\
        \hline
        J.Novak, A.Sladek, L.Willmeth & 10/26/17 & Initial document draft & 0.1 \\
        \hline
	    & & & \\
        \hline
    \end{tabular*}
\end{center}

\section{Project Overview}
\subsection{Introduction}
The Spaceport America Cup is an international engineering competition to design, build, and fly a student-made rocket to 30,000 feet.  The competition is scored on several criteria including software components like flight avionics, recording and displaying telemetry, and later displaying the results from an scientific payload.

\subsection{Purpose}
This document outlines the software requirements for the Spaceport America Cup 30k rocket competition in 2018.

\subsection{Scope}
The software described by this document will support the Oregon State University (OSU) American Institute of Aeronautics and Astronautics (AIAA) team's entry for the Spaceport America Cup 30k competition in the summer of 2018.

\subsection{Definitions, acronyms, and abbreviations}
\begin{center}
  \begin{tabular}{|l|l|}
      \hline
      AIAA	&American Institute of Aeronautics and Astronautics\\
      CS		&Computer Science\\
      \hline
      ECE		&Electrical and Computer Engineering\\
      GUI		&Graphical User Interface\\
      \hline
      GPIO	&General Purpose Input Output pin\\
      GPS		&Global Positioning System\\
      \hline
      OSU		&Oregon State University\\
      PCB		&Printed Circuit Board\\
      \hline
      SD card	&Secure Data card\\
      TBD		&To Be Determined\\
      \hline
  \end{tabular}
\end{center}

\subsection{References}
\subsection*{Initial Project Description}
http://eecs.oregonstate.edu/capstone/cs/capstone.cgi?project=340\\
Client Requirements Document assignment\\
http://eecs.oregonstate.edu/capstone/cs/capstone.cgi?hw=reqs\\
CS capstone group final Problem Statement\\
https://github.com/OregonStateRocketry/30k2018-CS-Capstone

\subsection{Overview}
This document contains the complete software requirements and specifications for the computer science portion of the Spaceport America Cup 30k rocket competition.  The software can be organized into four parts:

\begin{enumerate}
\item Ground station - \textbf{Parser program}, takes an audio source from a radio and stores APRS fields in a database.
\item Ground Station - \textbf{Display program}, reads from the database and displays information to multiple users.
\item \textbf{Rocket Avionics}, reads and logs onboard sensors, triggers flight events on the rocket.
\item \textbf{Payload Avionics}, reads and logs onboard sensors, triggers flight events on the payload.
\end{enumerate}

This document outlines the requirements for each of these programs organized as sets of user stories, characteristics, specific product functions, constraints, and early assumptions about program inputs and outputs. This document may be updated as the requirements change throughout the project.

\section{Overall description}
\subsection{Product perspective}
The Parser program will receive data from a file, or an external source (telemetry from the rocket) in the form of a radio signal. The specific data format will be agreed on by both the CS and ECE teams.  The Parser program will run on a ground station and parse out the individual fields, then store them in a database. The database may be hosted on the same computer, or another computer on the same network.

The Display program will run on a client computer and graph or otherwise display telemetry or payload data from the database.

The Rocket Avionics program will use onboard sensors to detect and trigger the separation of the rocket near the apogee of flight, as well as deploy the main parachute closer to the ground.  The optimal conditions for separation and deployment will be determined by team discussion and involving our experienced mentors.

The Payload Avionics program will use onboard sensors to signal a propeller to push the payload down in order to create a low gravity environment on the payload. It will also trigger deployment of a parachute before impacting the ground.  The optimal conditions to deploy the parachute will be determined by team discussion.

\subsubsection{System interfaces}
The Rocket Avionics software will interact with the rocket through the PCB designed by the ECE team.\\
The Payload Avionics software will interact with the payload through the PCB designed by the ECE team.\\
The Parser program will interact with an input source and database.\\
The Display program will interact with the same network and database as the Parser program.

\subsubsection{User interfaces}
The Rocket and Payload Avionics software will allow for the use of an LED to indicate status.

The Parser program will provide a text interface that can be used for general debugging purposes.

The user interfaces for our software will be limited. Users will be able to use the GUI to view recent telemetry data. They will also be able to read the graphs generated from the payload data to draw conclusions about its flight and experimental results.

\subsubsection{Hardware interfaces}
The Parser program will interact with a radio device to collect telemetry data.

The Rocket Avionics software will interact with the rocket through a PCB designed by the ECE team, which will provide sensors that will probably include at least one accelerometer, gyroscope, and barometer, as well as multiple explosive charges to separate the rocket, and deploy drogue and main parachutes.

The Payload Avionics software

\subsubsection{Software interfaces}
We may use external libraries or software to create graphs, especially from the data recorded on the scientific payload.  The parsing program will read and write to a local database.

(stretch goal) We will allow other computers or devices to connect to the computer running the parsing program, to view telemetry data and payload graphs via an intranet page.  This will require configuring wifi networks and generating dynamic graphs from a database.

\subsubsection{Communication interfaces}
Telemetry for the rocket will be transmitted as a radio signal.  The specific format and fields of the data will be determined by both the CS and ECE teams.

\subsubsection{Memory constraints}
Avionics software will need to execute on a computer system to be determined by the ECE team, which is likely to be a micro controller. All other software can be run on laptops with varied memory capacity.

\subsubsection{Operations}
The telemetry or analysis programs will allow the user to load data from a file, which will create static displays that do not change over time.

The telemetry program will also allow the user to monitor a radio signal as an input source, which will create a dynamic display as new data is received.

The avionics program does not offer a user interface because it operates solely based on sensor inputs.

\subsection{Product functions}
\subsubsection{Telemetry Program}
Import a data file from hard drive or SD card.\\
This can be used to either test a set of telemetry data, or display a previously completed flight.\\
Display one or more graphs or text fields of the telemetry data.

\subsubsection{Payload analysis}
Import a data file from hard drive or SD card.\\
This can be used to either test a set of telemetry data, or display a previously completed flight.\\
Display one or more graphs of the data recorded during flight.

\subsubsection{Avionics}
Read data from individual sensors\\
Toggle a GPIO pin to trigger rocket separation\\
Toggle a GPIO pin to trigger payload ejection\\
Toggle a GPIO pin to deploy drogue parachute (if needed)\\
Toggle a GPIO pin to deploy main parachute (if needed)

\subsection{User characteristics}
The intended users of this software will consist primarily of other team members and mentors. They will have technical knowledge related to operation and construction of the rocket and its sensors. These users are all seniors in college or beyond. The nature of the graphical data will be technical. The software team will also be present at launch to assist in the use of the software.

\subsection{Constraints}
\subsubsection{Regulatory policies}
The project may be subject to competition restrictions.\\
The radio transmissions must comply with federal communications commission regulations.

\subsubsection{Hardware limitations (e.g., signal timing requirements)}
The software must comply with the requirements of the computer hardware selected by the ECE team.\\
The parsing program requires an external radio to receive radio signals.\\
The parsing program may require a network connection to any connected clients.

\subsubsection{Interfaces to other applications}
The telemetry and analysis programs require an external data source.

\subsubsection{Parallel operation}
Multiple copies of the telemetry program can run simultaneously on different machines, without interfering with each other.

\subsubsection{Audit functions}
The telemetry and avionics programs will include a test suite covering at least 80\% of the CS team’s lines of code.

\subsubsection{Control functions}
Control functions are limited by the ECE and rocket hardware.

\subsubsection{Higher-order language requirements}
The telemetry software may be written in a high level language.\\
The avionics software will probably be written in C.

\subsubsection{Signal handshake protocols (e.g., XON-XOFF, ACK-NACK)}

\subsubsection{Reliability requirements}
GPS tracking must be reliable to ensure recovery of the rocket. Computations for display and parsing of data must be accurate.\\
Separation must be triggered at the right time consistently.

\subsubsection{Criticality of the application}
Displaying recent GPS data will be mission critical for recovery of the rocket. Interpretation and display of other data, including telemetry and payload is less critical, but still very important to the success of the launch. Separation at the right time is critical for recovery of the rocket.

\subsubsection{Safety and security considerations}
Successfully tracking the rocket and payload are mission-critical.\\
Successfully timing the separation of the rocket, and deploying the payload and both parachutes are mission-critical.

\subsection{Assumptions and dependencies}
Rocket will launch far enough away from spectators not to pose a safety threat.

The ECE team will provide the computer hardware necessary to run the avionics software and record data for use in the analysis program.

The ECE team will create a radio signal containing accurate and well-formatted telemetry data for both the rocket and payload components.

\section{Specific requirements}
\subsection{External interface requirements}
The parsing program will need to be connected to an input source capable of receiving a properly formatted radio signal.  Formatting may include audio signals in APRS format, as a serial string, or some other as-yet unknown but acceptable format.

\subsubsection{User interfaces}
The parsing and payload programs will include either text-based or graphical user interfaces.\\
The avionics program may or may not include a text-based interface.

\subsubsection{Hardware interfaces}
The avionics program will interface with a custom PCB designed and manufactured by the ECE team.
	
\subsubsection{Software interfaces}
The parsing and payload programs will be able to load a data file from an SD card or Linux operating system.

\subsubsection{Communications interfaces}
The parsing program will interface with the rocket through radio telemetry.

\subsection{System features}

\subsubsection{Parser}
The parser will interpret information received via radio signal. It will decode the message from an encoding to be determined by the ECE team and translated into a row to be added to a database. It will also check to ensure that the message received is actual telemetry data and possesses all of the required fields. If it appears to be invalid, not possessing the required fields, or inaccurate, falling vastly outside of the predicted scope in any field, it will be discarded. This would include a GPS location that falls outside the airspace of the test, a negative temperature, or an altitude that is several hundred feet below ground or over a hundred thousand feet above ground. If any field contains an unexpected character, the information will be discarded.

The parser will have a test suite with at least 80\% line coverage for the code the CS team writes.

The parser will be expected to push information to the database at regular enough intervals to ensure a five second delay from receipt of packet to display.

\subsubsection{Avionics}
Avionics programs will be written depending on the needs of other teams working on the rocket. They will likely be written in C or shell script, but this is dependant on the processor and sensors chosen by the ECE team. Avionics programs will be written for both the rocket and the payload, though responsibility for these will be split between the ECE team and our team.

\subsubsection{Graphs for Payload}
Telemetry data and experimental data for the payload will be graphed by our team. Outlier data will be identified using mathematical software. Fit lines for experimental data will be generated on request, and for all recorded telemetry. Detailed graphs will be generated for telemetry data.

\subsection{Performance requirements}
The parsing program should process incoming radio packets within 5 seconds of receiving them.  We expect to receive a GPS packet about once per second.

The payload program will generate graphs from an unknown amount of data, which makes a performance requirement difficult.

\subsection{Design constraints}
The parsing and payload programs will run on a linux operating system, requiring relatively inexpensive hardware that should cost no more than \$1,000.

The parsing, payload, and avionics programs will not require an active internet connection.

The avionics software will run on a processor to be determined by the ECE team.

\subsection{Software system attributes}
The parsing and payload programs will run on a linux operating system.

\subsection{Other requirements}


\end{document}