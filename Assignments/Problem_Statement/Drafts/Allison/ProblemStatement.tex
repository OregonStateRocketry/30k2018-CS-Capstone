\documentclass[onecolumn, draftclsnofoot,10pt, compsoc]{IEEEtran}
\usepackage{graphicx}
\usepackage{url}
\usepackage{setspace}

\usepackage{geometry}
\geometry{textheight=9.5in, textwidth=7in}

% 1. Fill in these details
\def \CapstoneTeamName{		CS 30K Avionics}
\def \CapstoneTeamNumber{		41}
\def \GroupMemberOne{			Allison Sladek}
\def \GroupMemberTwo{			Levi Willmeth}
\def \GroupMemberThree{			Joshua Novak}
\def \CapstoneProjectName{		30k Rocket Spaceport America}
\def \CapstoneSponsorCompany{	Oregon State Univesity}
\def \CapstoneSponsorPerson{		Nancy Squires}

% 2. Uncomment the appropriate line below so that the document type works
\def \DocType{		Problem Statement
				%Requirements Document
				%Technology Review
				%Design Document
				%Progress Report
				}

\newcommand{\NameSigPair}[1]{\par
\makebox[2.75in][r]{#1} \hfil 	\makebox[3.25in]{\makebox[2.25in]{\hrulefill} \hfill		\makebox[.75in]{\hrulefill}}
\par\vspace{-12pt} \textit{\tiny\noindent
\makebox[2.75in]{} \hfil		\makebox[3.25in]{\makebox[2.25in][r]{Signature} \hfill	\makebox[.75in][r]{Date}}}}
% 3. If the document is not to be signed, uncomment the RENEWcommand below
\renewcommand{\NameSigPair}[1]{#1}

%%%%%%%%%%%%%%%%%%%%%%%%%%%%%%%%%%%%%%%
\begin{document}
\begin{titlepage}
    \pagenumbering{gobble}
    \begin{singlespace}
    	\includegraphics[height=4cm]{coe_v_spot1}
        \hfill
        % 4. If you have a logo, use this includegraphics command to put it on the coversheet.
        %\includegraphics[height=4cm]{CompanyLogo}
        \par\vspace{.2in}
        \centering
        \scshape{
            \huge CS Capstone \DocType \par
            {\large\today}\par
            \vspace{.5in}
            \textbf{\Huge\CapstoneProjectName}\par
            \vfill
            %{\large Prepared for}\par
            %\Huge \CapstoneSponsorCompany\par
            %\vspace{5pt}
            %{\Large\NameSigPair{\CapstoneSponsorPerson}\par}
            {\large Prepared by }\par
            Group\CapstoneTeamNumber\par
            % 5. comment out the line below this one if you do not wish to name your team
            %\CapstoneTeamName\par
            \vspace{5pt}
            {\Large
                \NameSigPair{\GroupMemberOne}\par
                %\NameSigPair{\GroupMemberTwo}\par
                %\NameSigPair{\GroupMemberThree}\par
            }
            \vspace{20pt}
        }
        \begin{abstract}
        % 6. Fill in your abstract
        The purpose of the 30k Rocket Spaceport America project is to build a rocket that can reach an altitude of 30,000 feet.
        This rocket must be recoverable and achieve at least 10 seconds of microgravity in order to conduct a successful experiment as part of the payload challenge.
        As the computer science sub-group on the project, the main goal of this team will be to display telemetry and payload data, and display live GPS tracking of the rocket and payload.
        To ensure the rocket’s success, the team will sup- port software needs of other sub-groups, especially the avionics electrical engineering team.

        \end{abstract}
    \end{singlespace}
\end{titlepage}
\newpage
\pagenumbering{arabic}
\tableofcontents
% 7. uncomment this (if applicable). Consider adding a page break.
%\listoffigures
%\listoftables
\clearpage

% 8. now you write!
\section{Problem Statement}
The 30k Rocketry Challenge is an intercollegiate competition with participants aiming to build functional and recoverable rockets that reach 30,000.
The team is also entering a payload contest that will require an innovative experiment to take place in the air and after launch.
Oregon State’s team for this challenge consists of several sub-teams: aerodynamics and recovery, payload, propulsion, structures, avionics (electrical), and avionics (computer science).

\section{Proposed Solution}
The team will build software for a ground station that will manipulate and present the data recovered after the rocket retrieval as well as display live GPS location data for the rocket and payload.
Supporting the software need of other rocket sections will also be necessary, including detachment of payload in flight, and feedback for other teams while building and improving on the rocket.
The team will have to work closely with the electrical engineering avionics group.

\section{Performance Metrics}
The finished rocket must be recoverable.
This means that no parts are damaged to the point where a follow-up launch is not possible.
In order to recover the rocket, the team will be required to track the rocket with a live GPS signal.
The tracking software should provide clear location data and aid in recovery of the rocket.
Another primary goal of the team is to display collected data after the recovery of the rocket.
The data should be displayed in a clear and meaningful format that the other sub-teams can interpret and aid in their debugging process throughout the build process.
Data displayed is to include telemetry data collected by various rocket sensors and the microgravity experiment data from the payload.

\end{document}
